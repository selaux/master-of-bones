\documentclass[pdftex,12pt,a4paper]{report}
\usepackage{dbstmpl}
\usepackage{subfigure}
\usepackage[utf8]{inputenc}

% Hier die eigenen Daten eintragen
\global\arbeit{Masterarbeit}
\global\titel{Data Science in Zooarchaeology: Analyzing local shape variations in bone findings}
\global\bearbeiter{Stefan Lau}
\global\betreuer{Johannes Niedermayer}
\global\aufgabensteller{Dr. Matthias Renz}
\global\abgabetermin{31.08.2015}
\global\ort{München}
\global\fach{Medieninformatik}

\begin{document}

% Deckblatt
\deckblatt

% Erklaerung fuer das Pruefungsamt
\erklaerung

% Zusammenfassung
\begin{abstract}
Dieses Dokument dient als Muster f"ur die Ausarbeitung einer \the\arbeit\
an der Lehr- und Forschungseinheit f"ur Datenbanksysteme am Institut f"ur
Informatik der LMU M"unchen.
\end{abstract}

% Inhaltsverzeichnis
\tableofcontents

% Hier beginnt der eigentliche Text
\chapter{Introduction}

\chapter{Related Work}



\section{Morphometrics in Zooarcheology}

\cite{blackith1971multivariate}
\cite{adams2004geometric}
\cite{mitteroecker2009advances}

\section{Shape Matching in Data Science}

\cite{da2010shape}
\cite{veltkamp2001shape}
\cite{belongie2002shape}
\cite{mhamdi2014local}

\chapter{Mathematical Basics}

\section{Image Segmentation}

\cite{van2014scikit}

\section{Support Vector Machines}

\cite{pedregosa2011scikit}

\chapter{Detection of Local Shape Variations Inbetween Two Classes of Shapes}

\section{Overview}

\section{Preprocessing}

\subsection{Image Segmentation}

\cite{achanta2012slic}
\cite{felzenszwalb2004efficient}
\cite{houhou2009fast}
\cite{jain1990unsupervised}
\cite{haralick1973textural}

\subsection{Triangulation and Outlining}

\cite{akkirajualpha}

\subsection{Landmark Extraction}

\section{Shape Registration}

\section{Shape Comparison}

\subsection{Selection of Evaluation Windows}

\subsection{Feature Extraction}

\subsection{Dimensionality Reduction}

\subsection{Calculation of Separability Measures}

\chapter{Detection of Differences Inbetween Two Classes of Bones That Are Measurable On-Site}

\section{Definition of Measurable Values}

\section{Training Decision Trees}

\chapter{Experiments}

\section{Tests on Real-World Data}

\subsection{Results and Compliance with Former Findings}

\subsection{Comparison of Algorithm Parameters}

\subsection{Comparison of Features}

\subsection{Comparison of Registration Strategies}

\subsection{Comparison of Separability Measures}

\section{Tests on Synthetic Data}

\subsection{Generating Synthetic Data}

\subsection{Influence of Variances in Shape Features}

\subsection{Influence of Errors in Shape Measurement}

\subsection{Influence of Errors in Shape Registration}

\chapter{Conclusion}

\section{Summary}

\section{Discussion}

\section{Future Work}

% Abbildungsverzeichnis (kann auch nach dem Inhaltsverzeichnis kommen)
\listoffigures

% Tabellenverzeichnis (kann auch nach dem Inhaltsverzeichnis kommen)
\listoftables

% Literaturverzeichnis
\bibliographystyle{dbstmpl}    % verwendet dbstmpl.bst
% alternative, vorinstallierte Stile sind z.B. plain oder abbrv
\bibliography{dbstmpl}         % verwendet dbstmpl.bib

\end{document}
