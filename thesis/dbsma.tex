\documentclass[pdftex,12pt,a4paper]{report}
\usepackage{dbstmpl}
\usepackage{subfigure}
\usepackage[utf8]{inputenc}

% Hier die eigenen Daten eintragen
\global\arbeit{Masterarbeit}
\global\titel{Data Science in Zooarchaeology: Analyzing local shape variations in bone findings}
\global\bearbeiter{Stefan Lau}
\global\betreuer{Johannes Niedermayer}
\global\aufgabensteller{Dr. Matthias Renz}
\global\abgabetermin{31.08.2015}
\global\ort{München}
\global\fach{Medieninformatik}

\begin{document}

% Deckblatt
\deckblatt

% Erklaerung fuer das Pruefungsamt
\erklaerung

% Zusammenfassung
\begin{abstract}
Dieses Dokument dient als Muster f"ur die Ausarbeitung einer \the\arbeit\
an der Lehr- und Forschungseinheit f"ur Datenbanksysteme am Institut f"ur
Informatik der LMU M"unchen.
\end{abstract}

% Inhaltsverzeichnis
\tableofcontents

% Hier beginnt der eigentliche Text
\chapter{Introduction}

\chapter{Problem Definition}

\chapter{Related Work}

\section{Morphometrics in Zooarcheology}

\cite{blackith1971multivariate}
\cite{adams2004geometric}
\cite{mitteroecker2009advances}

\section{Shape Matching in Data Science}

\cite{da2010shape}
\cite{veltkamp2001shape}
\cite{belongie2002shape}
\cite{mhamdi2014local}

\chapter{Mathematical Basics}

\section{Image Segmentation}

\cite{van2014scikit}

\section{Support Vector Machines}

\cite{pedregosa2011scikit}

\section{Decision Trees}

\chapter{Detection of Local Shape Variations}

\section{Overview}

\section{Preprocessing}

\subsection{Image Segmentation}

\cite{achanta2012slic}
\cite{felzenszwalb2004efficient}
\cite{houhou2009fast}
\cite{jain1990unsupervised}
\cite{haralick1973textural}

\subsection{Triangulation and Outlining}

\cite{akkirajualpha}

\subsection{Landmark Extraction}

As some of the features we propose in this work are landmark based, and the detection of on-site-measurable differences
is based on landmarks as well, a method to extract these landmarks from the outline of the bone is needed. In contrast to
our approach, former work on the talus by zooarcheologists was based on manually located landmarks that correspond to
biological features of the bone. Since our goal was to automate the process as much as possible and some of the landmarks
are not positioned on the outline of the bone, this was not feasible in our case. We selected landmarks on the outline
of the bone that could be located automatically and overlap as much as possible with the landmark definitions by the
zooarcheologists. For this purpose we introduced two methods that extract landmarks from outlines of Tali. The landmarks
found by these methods have very similar positions but differ in on-site-measurability and stability in-between classes.
Figure \ref{figure-landmark-methods} shows the results of the three landmark locating methods: manual, with angles and
with space partitioning. The biological correspondences of the landmarks are as follows.

\begin{itemize}
\item Landmark 1: Highest point on the lateral crest
\item Landmark 2: Lowest point in between the crests
\item Landmark 3: Highest point on the medial crest
\item Landmark 4: Intersection between the medial outer edge and the medially arcing edge of the medial crest
\item Landmark 5: Outmost point on the medial apex
\item Landmark 6: Lowest point on the medial outer edge
\item Landmark 7: Highest point on the distal depression
\item Landmark 9: Intersection of the distal articular surface with the Calcaneus
\item Landmark 10: Lateral end point of the lower muscle attachment line
\item Landmark 11: Intersection of the two muscle attachment lines
\item Landmark 12: Proximal end point of the lateral protrusion on the lateral crest
\end{itemize}

The landmarks that were extracted automatically are landmarks 1, 2, 3, 5, 6, 7. Furthermore we introduced the new
landmark 8, which has no biological correspondence, but can be located robustly on all bone outlines.

\subsubsection{Extraction by Angle}

When extracting a landmark by angle, the landmark position is defined by a minimum angle $\alpha_{min}$, a maximum
angle $\alpha_{max}$ and the information wether it is located on a minimum or a maximum turning point.
The minimum/maximum turning angle $\alpha_{t}$ between $\alpha_{min}$ and $\alpha_{max}$ of the following function
is calculated, where $i(\alpha)$ is the intersection between a ray from the coordinate center with the angle $\alpha$ and
the outline. $i(\alpha_t)$ is then used as the landmark.

\begin{equation}
d(\alpha) = ||i(\alpha)||
\end{equation}

The definitions for all landmarks when using this method can be found in Appendix \ref{appendix:landmarks-angle}.

The landmarks found by this method are robustly located across all classes, but have a distinct disadvantage. They are
not as easily located by hand since they dont lie exactly on the high  points of the outline, but slightly beneath
them  as seen in Figure \ref{figure-landmark-methods}. This makes distances inbetween landmarks harder to measure, which
will become relevant in Chapter \ref{chapter:measurable-differences}.

\subsubsection{Extraction by Space Partitioning}

\section{Shape Registration}

\section{Shape Comparison}

\subsection{Selection of Evaluation Windows}

\subsection{Feature Extraction}

\subsection{Dimensionality Reduction}

\subsection{Calculation of Separability Measures}

\chapter{Detection of Differences That Are Measurable On-Site}
\label{chapter:measurable-differences}

\section{Definition of Measurable Values}

\section{Training Decision Trees}

\chapter{Experiments}

\section{Tests on Real-World Data}

\subsection{Results and Compliance with Former Findings}

\subsection{Comparison of Algorithm Parameters}

\subsection{Comparison of Features}

\subsection{Comparison of Registration Strategies}

\subsection{Comparison of Separability Measures}

\section{Tests on Synthetic Data}

\subsection{Generating Synthetic Data}

\subsection{Influence of Variances in Shape Features}

\subsection{Influence of Errors in Shape Measurement}

\subsection{Influence of Errors in Shape Registration}

\chapter{Conclusion}

\section{Summary}

\section{Discussion}

\section{Future Work}

% Abbildungsverzeichnis (kann auch nach dem Inhaltsverzeichnis kommen)
\listoffigures

% Tabellenverzeichnis (kann auch nach dem Inhaltsverzeichnis kommen)
\listoftables

% Literaturverzeichnis
\bibliographystyle{dbstmpl}    % verwendet dbstmpl.bst
% alternative, vorinstallierte Stile sind z.B. plain oder abbrv
\bibliography{dbstmpl}         % verwendet dbstmpl.bib

\end{document}
