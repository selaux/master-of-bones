\documentclass[pdftex,12pt,a4paper]{report}
\usepackage{dbstmpl}
\usepackage{subfigure}
\usepackage[utf8]{inputenc}

% Hier die eigenen Daten eintragen
\global\arbeit{Masterarbeit}
\global\titel{Data Science in Zooarchaeology: Analyzing local shape variations in bone findings}
\global\bearbeiter{Stefan Lau}
\global\betreuer{Johannes Niedermayer}
\global\aufgabensteller{Dr. Matthias Renz}
\global\abgabetermin{31.08.2015}
\global\ort{München}
\global\fach{Medieninformatik}

\begin{document}

% Deckblatt
\deckblatt

% Erklaerung fuer das Pruefungsamt
\erklaerung

% Zusammenfassung
\begin{abstract}
Dieses Dokument dient als Muster f"ur die Ausarbeitung einer \the\arbeit\
an der Lehr- und Forschungseinheit f"ur Datenbanksysteme am Institut f"ur
Informatik der LMU M"unchen.
\end{abstract}

% Inhaltsverzeichnis
\tableofcontents

% Hier beginnt der eigentliche Text
\chapter{Ein Kapitel}

Das ist die Ebene {\tt chapter}.

\section{Ein Abschnitt}

Das ist die Ebene {\tt section}.

\subsection{Ein Unterabschnitt}

Das ist die Ebene {\tt subsection}.

\subsubsection{Ein Unterunterabschnitt}
\label{sec:a}

Das ist die Ebene {\tt subsubsection}.

\subsubsection{Noch ein Unterunterabschnitt}
\label{sec:b}

Wer \ref{sec:a} sagt, mu"s auch \ref{sec:b} sagen.

\subsection{Noch ein Unterabschnitt}

Das ist ein gew"ohnlicher Absatz.

\paragraph{Ein Absatz mit Titel}

Das ist die Ebene {\tt paragraph}.

\subparagraph{Ein Unterabsatz mit Titel}

Das ist die Ebene {\tt subparagraph}.

\subsection*{Ein nicht numerierter Unterabschnitt}

Dieser Unterabscnitt erscheint nicht im Inhaltsverzeichnis.

\section{Beispiele}
Dieses Beispieldokument wurde mit dem \LaTeX-Paket {\tt dbstmpl.sty} erzeugt.
{\tt dbstmpl.sty} definiert Befehle zum Erzeugen von einigen
h"aufig gebrauchten Symbolen.

\begin{align}
  \N \subset \Z \subset \Q \subset \R \subset \C
\end{align}

Weitere Symbole k\"onnen im {\tt symbols-a4.pdf}-Dokument, das f\"ur
gew\"ohnlich bei \LaTeX-Distributionen mitgeliefert wird, nachgeschlagen
werden. Alternativ gibt es diverse Hilfeseiten im Netz.

Au"serdem werden automatisch einige zus"atzliche Pakete geladen.
Mit dem Paket {\tt graphicx} k"onnen Grafiken eingebunden werden.
Abbildung~\ref{fig:dbslogo} zeigt das Logo der LFE f"ur Datenbanksysteme.

\begin{figure}
  \begin{center}
    \includegraphics[width=2cm]{logo-dbs.png}
  \end{center}
  \caption{Altes Logo der LFE f"ur Datenbanksysteme.
    \label{fig:dbslogo} % Figures und Tables können gelabelt werden.
    % Es empfiehlt sich, dies in der Caption zu machen
  }
\end{figure}

{\tt amsmath} erlaubt die bequeme Alignierung von mehrzeiligen Formeln:

\begin{align}
  f(a,b) &= (a-b)^2 \label{eq:aMinusbSq}\\
    &= a^2 - 2 a b + b^2 \label{eq:aMinusbSqResolved}
\end{align}

Zus\"atzlich k\"onnen diese Formeln auch referenziert werden. In~\eqref{eq:aMinusbSq}
wird das Problem gestellt und in~\eqref{eq:aMinusbSqResolved} wird es etwas
aufgel\"ost. Nat\"urlich k\"onnen die Gleichungen auch ohne Klammern
referenziert werden: \ref{eq:aMinusbSq}, \ref{eq:aMinusbSqResolved}.

Das in diesem Dokument geladene Paket {\tt subfigure} (oben eingebunden durch
\verb=\usepackage{subfigure}=) erlaubt das Erzeugen von Unterabbildungen.
Abbildung~\ref{fig:logos} enth"alt zwei Unterabbildungen.
Abbildung~\ref{fig:logos-a} zeigt nochmal das Logo der LFE f"ur
Datenbanksysteme, \ref{fig:logos-b} zeigt das Logo der
Ludwig-Maximilians-Universit"at.

\begin{figure}
  \begin{center}
    \subfigure[Altes DBS-Logo.\label{fig:logos-a}]{
      \makebox[0.48\textwidth]{
        \includegraphics[width=2cm]{logo-dbs.png}}}
    \hspace{\fill}
    \subfigure[Uraltes LMU-Logo.\label{fig:logos-b}]{
      \makebox[0.48\textwidth]{
        \includegraphics[width=5cm]{logo-lmu.png}}}
  \end{center}
  \caption{Zwei Logos.}
  \label{fig:logos}
\end{figure}

Tabelle~\ref{tab:quadratzahlen} zeigt die ersten f"unf Quadratzahlen.

\begin{table}
  \begin{center}
  \begin{tabular}{|r|r|}
    \hline \makebox[1cm]{$x$} & \makebox[1cm]{$x^2$}\\
    \hline\hline 1 &  1\\
    \hline 2 &  4\\
    \hline 3 &  9\\
    \hline 4 & 16\\
    \hline 5 & 25\\
    \hline
  \end{tabular}
  \end{center}
  \caption{Die Zahlen 1 bis 5 und ihre Quadrate.}
  \label{tab:quadratzahlen}
\end{table}

Folgenderma\ss en wird Literatur referenziert.
DBSCAN~\cite{EKSX96} und OPTICS~\cite{ABKS99} sind Beispiele f\"ur
dichtebasierte Clusteringverfahren. Diese Eintr\"age werden als
{\tt bibitem} im {\tt .bbl}-File referenziert. Um dieses File braucht man
sich erfreulicherweise nicht zu k\"ummern, weil es automatisch aus einer
Sammlung von BibTeX-Eintr\"agen in einem Literatur-File gebildet werden
kann. Hier ist wichtig, dass dieses File den BibTeX-Konventionen
\footnote{{\tt http://www.tex.ac.uk/tex-archive/biblio/bibtex/contrib/doc/btxdoc.pdf}}
entspricht, sonst kommen kreative Fehlermeldungen. Daf\"ur braucht es
besagte Literatur-Sammlung -- hier {\tt dbstmpl.bib}
und einen Literatur-Stil, den man auch selber definieren kann, wie
etwa {\tt dbstmpl.bst}.

\section{Bemerkungen}

Eine "Ubersicht "uber alle Abbildungen und Tabellen einer Arbeit
verschaffen das Abbildungs- und das Tabellenverzeichnis. Diese
sollten entweder nach dem Inhaltsverzeichnis oder (wie in diesem
Dokument) vor dem Literaturverzeichnis eingef"ugt werden.
Je nach Bedarf und Umfang der Arbeit k"onnen auch Verzeichnisse
f"ur Definitionen, S"atze oder Lemmata angelegt oder nicht
ben"otigte Verzeichnisse weggelassen werden.

\appendix

\chapter{Ein Anhang}

Im Anhang kann auf Implementierungsaspekte wie Datenbankschemata
oder Programmcode eingegangen werden.

\section{\protect{pdflatex} vs.\ \protect{latex}}

Dieses Dokument wird mit {\tt pdflatex} gebaut. Das ist n\"otig, wenn
wie hier Bilder im {\tt .jpg}, {\tt .png} oder {\tt .pdf}-Format eingebunden
werden. Alternativ kann auch alles \"uber das klassische \LaTeX-Kommando
laufen: \verb=latex dbsba=. Damit wird ein {\tt .dvi} erstellt, welches
erst mit {\tt dvipdf} zu {\tt .pdf} konvertiert werden muss. Dies
bedeutet, dass Bilder nur noch als {\tt .eps} eingebunden werden
k\"onnen. {\em Merke: {\tt .pdf}-affine und {\tt .eps}-Bilder k\"onnen
NICHT kombiniert werden, also entweder oder.}

\section{Ich will aber englisch!}

Ihr d\"urft auch gerne englische \the\arbeit en schreiben. Daf\"ur m\"usst
ihr im Stylefile {\tt dbstmpl.sty} die Zeile
\verb=\RequirePackage[english,german]{babel}= durch
\verb=\RequirePackage[german,english]{babel}= austauschen. Dann hei\ss en
Chapter wieder Chapter und nicht Kapitel, etc.

\section{Ich h\"atte da einen eigenen Stil.}

Ihr k\"onnt auch eigene Stile definieren und verwenden, wenn er die
Informationen des Deckblatts und die darauffolgende Erkl\"arung
enth\"alt, sowie die Gnade des Betreuers gefunden hat (es
empfiehlt sich also, fr\"uhzeitg kurz nachzufragen).

\section{Hilfe?}

Gibt es online zuhauf:

\begin{description}
  \item[Das \LaTeX Kochbuch]~\\{\tt http://www.uni-giessen.de/hrz/tex/cookbook/cookbook.html}
  \item[CTAN] {\tt http://www.ctan.org/} zum manuellen Download von Stylefiles
      und deren Dokumentation
  \item[Diverse Mathe-Tweaks]~\\ \verb=http://meta.wikimedia.org/wiki/Help:Displaying_a_formula=
  \item[B\"uchersuche mit BibTeX -Unterst\"utzung] {\tt http://lead.to/amazon/en/}
  \item[JabRef] Java-basierter BibTeX -Manager {\tt http://jabref.sourceforge.net/}
  \item[FAQs] {\tt http://www.tex.ac.uk/cgi-bin/texfaq2html}
\end{description}

Und zur Not versteht auch euer Betreuer vermutlich was davon.


% Abbildungsverzeichnis (kann auch nach dem Inhaltsverzeichnis kommen)
\listoffigures

% Tabellenverzeichnis (kann auch nach dem Inhaltsverzeichnis kommen)
\listoftables

% Literaturverzeichnis
\bibliographystyle{dbstmpl}    % verwendet dbstmpl.bst
% alternative, vorinstallierte Stile sind z.B. plain oder abbrv
\bibliography{dbstmpl}         % verwendet dbstmpl.bib

\end{document}
